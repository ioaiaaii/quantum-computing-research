\section*{Teleportation}

Teleportation is a communication protocol that transfers quantum states between two ends. In Classical Systems, we can copy a state and easily encode, transfer, and decode it from one end to another. Quantum Systems and Quantum Mechanics work differently. Using a classical methodology to transfer a Qubit state is impossible due to the no-cloning theorem. In addition, if we measure a Qubit State, which could be in superposition, and then transfer the classical collapsed state, we do not improve network communication. Thus, Teleportation uses the Quantum Mechanical principle of Entanglement, which acts a communication network without physical subset. Entanglement creates an entangled state between quantum systems, in which the state cannot be decomposed or expressed as a tensor product of the individual states, hence the measurement of one affects the measurement of others. Since this can be done from a distance, as an instant action, the length of the distance is irrelevant to this affection; it is also called non-local. Thus, using Entanglement, Teleportation can move a quantum state from one end to another without physically transferring anything. However, since Entangled states are equally distributed probabilistic states, the information remains saved but not accessible (at least from human classical entities). Thus, teleportation has a final step in the protocol, which is to measure the sent qubit and communicate its classical state to the other end. The final end acts based on the received classical states of the measured entangled state and the teleported qubit state to decode the state into a new qubit. Thus, the qubit entity no longer exists in the sender part but only in the recipient part, which respects the non-cloning theorem. Finally, since additional classical information has to be sent via classical networks, which cannot travel FTL, Teleportation and Entanglement do not violate special relativity. 
